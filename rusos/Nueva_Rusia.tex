\bigskip 
\bigskip 
 
\subsubsection{Contenido}
\bigskip 
\begin{description}
\item[Nombre:] Nueva Rusia
\item[Cristaleria:] Shot (3oz / 90cc)
\item[M\'etodo de elaboraci\'on:] Directo
\item[Decoraci\'on:] Sin decoración.
\end{description}

\begin{table}[h]
\caption{Ingredientes y proporciones}
\label{tab:fonts}
\begin{center}       
\begin{tabular}{|l|l|l|c|l|} %% this creates two columns
%% |l|l| to left justify each column entry
%% |c|c| to center each column entry
%% use of \rule[]{}{} below opens up each row
\hline
\rule[-1ex]{0pt}{3.5ex}  \textbf{Producto} & \textbf{Bebida} & \textbf{Marca} & \textbf{Volumen} & \textbf{Fraccion}  \\
\hline
\rule[-1ex]{0pt}{3.5ex}  Jarabe & Granadina		& Cusenier 	& 1 oz / 30 cc 	&  	\\
\hline
\rule[-1ex]{0pt}{3.5ex}  Licor		& Blue Curacao 	& Bols		& 1 oz / 30 cc 	& 	\\
\hline
\rule[-1ex]{0pt}{3.5ex}  Aguardiente		& Vodka 	& Wyborowa		& 1 oz / 30 cc 	& 	\\

\end{tabular}
\end{center}
\end{table} 
\bigskip 

%%-----------------------------------------------------------
\subsubsection{Formato de elaboraci\'on} 
\label{sec:title}
\bigskip 
\begin{center}
\begin{enumerate}
\item Servir en un vaso de shot los ingredientes.
\item Es muy importante respetar el orden de los mismos, ya que la idea del trago es formar la bandera rusa en el shot.
\item Servir primero la granadina, luego, lentamente el blue curacao, y finalmente el vodka.
\end{enumerate}
\end{center}
\bigskip \bigskip 
\bigskip 
%%%%%%%%%%%%%%%%%%%%%%%%%%%%%%%%%%%%%%%%%%%%%%%%%%%%%%%%%%%%%

\subsubsection{Notas}
\bigskip 
\begin{center}
\raggedright{}Servir sin sorbete.
\end{center} 
\bigskip \bigskip \bigskip 

\end{center}