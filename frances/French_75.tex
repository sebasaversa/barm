\bigskip 
\bigskip 
 
\subsubsection{Contenido}
\bigskip 
\begin{description}
\item[Nombre:] French 75
\item[Cristaleria:] Vaso de champagne (5oz / 150cc)
\item[M\'etodo de elaboraci\'on:] Batido
\item[Decoraci\'on:] C\'ascara de lim\'on cortada en espiral
\end{description}

\begin{table}[h]
\caption{Ingredientes y proporciones} 
\label{tab:fonts}
\begin{center}       
\begin{tabular}{|l|l|l|c|l|} %% this creates two columns
%% |l|l| to left justify each column entry
%% |c|c| to center each column entry
%% use of \rule[]{}{} below opens up each row
\hline
\rule[-1ex]{0pt}{3.5ex}  \textbf{Producto} & \textbf{Bebida} & \textbf{Marca} & \textbf{Volumen} & \textbf{Fracci\'on}  \\
\hline
\rule[-1ex]{0pt}{3.5ex}  Aguardiente & Gin 			& Boodles British 		& 1 $\frac{1}{2}$ oz / 45 cc 	&  	\\
\hline
\rule[-1ex]{0pt}{3.5ex}  Espumante	& Champage Brut 	& Cristal 				& 3 oz / 80 cc 		&  	\\
\hline
\rule[-1ex]{0pt}{3.5ex}  Fruta 		& Lim\'on	 	& Exprimido				& $\frac{1}{2}$	oz		& 	\\
\hline
\rule[-1ex]{0pt}{3.5ex}  Fruta 		& Lim\'on	 	& C\'ascara				& ANP			& 	\\
\hline
\rule[-1ex]{0pt}{3.5ex}  Az\'ucar 		& 			 	& 					& 1 cdita		& 	\\
\hline
\end{tabular}
\end{center}
\end{table} 
1 cdita de az\'ucar = $\frac{1}{2}$ oz = 5gr.
\bigskip 

%%-----------------------------------------------------------
\subsubsection{Formato de elaboraci\'on} 
\label{sec:title}
\bigskip 
\begin{center}
\begin{enumerate}
\item Llenar un vaso de champagne con hielo.
\item Colocar hielo picado en una coctelera.
\item Colocar el gin, el lim\'on y el champagne en la coctelera y batir.
\item Colar y servir en el vaso del paso 1.
\item Decorar el vaso con la c\'ascara de lim\'on cortada en espiral.
\end{enumerate}
\end{center}
\bigskip \bigskip 
\bigskip 
%%%%%%%%%%%%%%%%%%%%%%%%%%%%%%%%%%%%%%%%%%%%%%%%%%%%%%%%%%%%%

\subsubsection{Notas}
\bigskip 
\begin{center}
\raggedright{}Servir sin sorbete.
\end{center} 
\bigskip \bigskip \bigskip 
\subsubsection{Informaci\'on extra}
\bigskip
\begin{center}

\medskip 
\raggedright{ \textbf{Or\'igenes de este trago}} \\ 
\medskip

{\justifying{
Este trago cuenta con diversas versiones de su origen. Entre ellas: \\

\medskip

\textbf{Primera historia} \\

La bebida fue creada en 1915 en el Harry’s New York Bar de Par\'is, por el barman mundialmente conocido, Harry MacElhone, es muy probable que en el Harry’s New York Bar, fuera donde el coñac se reemplazara por la ginebra, debido a que el bar, el m\'as emblem\'atico del Par\'is de principios del siglo pasado, era el lugar preferido de la American Field Ambulance Service Corps, y de otros muchos estadounidenses que por este tiempo viv\'ian en Par\'is, entre los que cabe destacar a los escritores Francis Scott Fitzgerald y Ernest Hemingway, ambos reconocidos aficionados de la cocteler\'ia. M\'as tarde el c\'octel French 75 llegar\'ia a los Estados Unidos y fue popularizado en el Stork Club de Nueva York.

\bigskip

\textbf{Segunda historia} \\

\indent La bebida fue creada durante la Primera Guerra Mundial originalmente por Raoul Lufbery, un famoso piloto franco-estadounidense. Lufbery pertenec\'ia a la Escadrille Am\'ericaine, tambi\'en conocida como la Escuadrilla Lafayette. Cuenta la leyenda que a Raoul Lufbery le gustaba el champagne pero despu\'es de las misiones de combate necesitaba algo m\'as potente, por lo que decidi\'o añadirle co\~nac, ya que este ingrediente en Francia era f\'acil de conseguir. \\
\indent El c\'octel era tan fuerte que supuestamente ten\'ia tanta potencia como los proyectiles que lanzaba el famoso ca\~n\'on de campa\~na M1897 de 75 mm, orgullo de la artiller\'ia de campa\~na francesa durante la Primera Guerra Mundial. El ca\~n\'on de tiro r\'apido y con un mecanismo de retroceso hidroneum\'atico, en su tiempo la pieza m\'as moderna de la artillería francesa, tambi\'en era conocido como “Franc\'es 75” o “Soixante Quinze” en franc\'es, de ah\'i que Raoul Lufbery decidiera bautizar al c\'octel en cuesti\'o como Franc\'es 75, en honor a la sofisticada pieza de artiller\'ia del ejercito franc\'es.

}\par}
\end{center}