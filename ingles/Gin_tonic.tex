\bigskip 
\bigskip 
 
\subsubsection{Contenido}
\bigskip 
\begin{description}
\item[Nombre:] Gin \& Tonic
\item[Cristaleria:] Rock glass (6oz / 180cc)
\item[M\'etodo de elaboraci\'on:] Directo
\item[Decoraci\'on:] Rodajas de lima
\end{description}

\begin{table}[h]
\caption{Ingredientes y proporciones} 
\label{tab:fonts}
\begin{center}       
\begin{tabular}{|l|l|l|c|l|} %% this creates two columns
%% |l|l| to left justify each column entry
%% |c|c| to center each column entry
%% use of \rule[]{}{} below opens up each row
\hline
\rule[-1ex]{0pt}{3.5ex}  \textbf{Producto} & \textbf{Bebida} & \textbf{Marca} & \textbf{Volumen} & \textbf{Fracci\'on}  \\
\hline
\rule[-1ex]{0pt}{3.5ex}  Aguardiente & Gin 			& Boodles British 		& 2 oz / 60 cc 	&  	\\
\hline
\rule[-1ex]{0pt}{3.5ex}  Gaseosa		& Agua t\'onica 	& Schweppes 				& 4 oz / 120 cc 		&  	\\
\hline
\rule[-1ex]{0pt}{3.5ex}  Fruta 		& Lim\'on	 	& Exprimido				& $\frac{1}{2}$	lim\'on		& 	\\
\hline
\rule[-1ex]{0pt}{3.5ex}  Fruta 		& Lima		 	& 						& 1 rodaja		& 	\\
\hline
\end{tabular}
\end{center}
\end{table} 
\bigskip 

%%-----------------------------------------------------------
\subsubsection{Formato de elaboraci\'on} 
\label{sec:title}
\bigskip 
\begin{center}
\begin{enumerate}
\item Llenar un vaso Rock Glass con hielo.
\item Exprimir medio lim\'on en el vaso.
\item A\~nadir el gin y completar con el agua t\'onica.
\item Decorar con la rodaja de lima.
\end{enumerate}
\end{center}
\bigskip \bigskip 
\bigskip 
%%%%%%%%%%%%%%%%%%%%%%%%%%%%%%%%%%%%%%%%%%%%%%%%%%%%%%%%%%%%%

\subsubsection{Notas}
\bigskip 
\begin{center}
\raggedright{}Servir con sorbete.
\end{center} 
\bigskip \bigskip \bigskip 
\subsubsection{Informaci\'on extra}
\bigskip
\begin{center}

\medskip 
\raggedright{ \textbf{Or\'igenes de este trago}} \\ 
\medskip

{\justifying{
El origen del gin-tonic se sit\'ua en India, en el siglo XIX. Fue en la \'epoca, de expansi\'on territorial de Inglaterra. Durante su paso por India, donde predominaba la malaria, los colonos brit\'anicos tomaban quinina para evitar contagiarse. Preparaban una mezcla con quinina, agua y aromatizantes. M\'as tarde, Cadbury Schweppes (conocida hoy en d\'ia por sus bebidas gasificadas) sustituy\'o el agua por soda, para hacerla m\'as digerible, y crearon as\'i la ''Indian Water Tonic''. \\
\indent Para hacer la bebida m\'as rica, los soldados brit\'anicos, le a\~{n}adieron alcohol: la ginebra Bombay destilada en la ciudad del mismo nombre. As\'i nace el gin-tonic. En poco tiempo la mezcla pas\'o de ser una medicina preventiva a una bebida refrescante y social hasta el punto que en el a\~no 1870 la mezcla de Indian Water Tonic con ginebra era m\'as que conocida.\\ 
\indent Actualmente es una de las bebidas m\'as conocidas y consumidas del mundo y la podemos encontrar en cualquier bar al que vayamos.

La quinina es un alcaloide extra\'ido de la corteza del Quino. Es de sabor muy amargo con propiedades curativas. Anti\"{u}amente era utilizado para tratar la malaria, hasta que fue reemplazado por medicamentos sint\'eticos m\'as eficaces.
}\par}
\end{center}